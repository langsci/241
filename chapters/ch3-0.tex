\section*{Présentation}

Cette troisième partie est consacrée à la façon dont les signes linguistiques se combinent pour former des syntagmes et aux structures qui en découlent. Le premier chapitre, où nous donnons notre définition de la syntaxe, est consacré à distinguer l’unité minimale de la syntaxe, le syntaxème, du syntagme. Le deuxième chapitre introduit la structure de connexion en montrant comment la combinaison des unités syntaxiques définit un ensemble de connexions qui forment la charpente de la structure syntaxique. Le troisième chapitre montre que cette structure est hiérarchique en introduisant les notions de tête et de dépendance. Le quatrième chapitre étudie les équivalences et différences entre les structure de dépendance, qui mettent en avant les relation syntagmatiques entre unités et les arbres de constituants, qui mettent en avant les relations d’enchâssement entre les unités. Le cinquième chapitre étudie l’ordre des mots et la façon dont les syntaxèmes se regroupent lorsqu’ils sont dans l’ordre linéaire, définissant ainsi ce que nous appelons la structure topologique.
